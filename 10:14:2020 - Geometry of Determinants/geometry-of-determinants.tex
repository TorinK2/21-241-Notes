\documentclass[12pt]{amsart}

\addtolength{\hoffset}{-2.25cm}
\addtolength{\textwidth}{4.5cm}
\addtolength{\voffset}{-2.5cm}
\addtolength{\textheight}{5cm}
\setlength{\parskip}{0pt}
\setlength{\parindent}{15pt}

\usepackage{amsthm}
\usepackage{amsmath}
\usepackage{amssymb}
\usepackage{xfrac}
\usepackage[colorlinks = true, linkcolor = black, citecolor = black, final]{hyperref}

\usepackage{graphicx}
\usepackage{multicol}
\usepackage{ marvosym }
\usepackage{wasysym}
\newcommand{\ds}{\displaystyle}


\pagestyle{myheadings}

\setlength{\parindent}{0in}

\pagestyle{empty}

\begin{document}

\thispagestyle{empty}

{\scshape 21-241} \hfill {\scshape \Large Notes} \hfill {\scshape Fall 2020}
\medskip
\hrule
\bigskip

\section*{Geometry of Determinants (10/14/2020)}
A determinant is a measurement of $signed$ $volume$. We will investigate this by calculating linear transformations at looking at geometric results. When thinking about a linear transformation (i.e. matrix multiplication), the basis is important for determining what is going to happen! Any linear transformation of a subspace can be generalized to a linear transformation of the bases of the subspace.\\
\\
\textbf{An Example: }\\Consider $A=\begin{pmatrix}a&b\\c&d\end{pmatrix}$. This matrix via multiplication (i.e. $A\vec{x}=\vec{y}$) will map vectors from $\mathbb{R}^2$ to $\mathbb{R}^2$. We can use some simple matrices to get rows and columns!
\begin{itemize}
	\item $\begin{pmatrix}a&b\\c&d\end{pmatrix}\begin{pmatrix}1\\0\end{pmatrix}=\begin{pmatrix}a\\c\end{pmatrix}$
	\item $\begin{pmatrix}a&b\\c&d\end{pmatrix}\begin{pmatrix}0\\1\end{pmatrix}=\begin{pmatrix}b\\d\end{pmatrix}$
	\item $\begin{pmatrix}a&b\\c&d\end{pmatrix}\begin{pmatrix}0\\1\end{pmatrix}= A(x\begin{pmatrix}1\\0\end{pmatrix} + y\begin{pmatrix}0\\1\end{pmatrix}) = x\begin{pmatrix}a\\c\end{pmatrix} + y\begin{pmatrix}b\\d\end{pmatrix}=\begin{pmatrix}ax+by\\cx+dy\end{pmatrix}$
\end{itemize}
The important thing to note here is that for any $(x, y)$, multiplication to $A$ is really a linear transformation of the columns of $A$. Accordingly, this is going to form a parallelogram quadrilateral of vectors with points of $(0,0)$, $(b,d)y$, $(a,c)x$, and finally $(a,c)x + (b,d)y$.\\
\\
\textbf{Thinking About Area: }\\
So, based on that values in our $2\times2$ matrix $A$, what is the area going to be? Well, he have a parallelogram with the points listed above. We can use some geometry to subscribe the parallelogram into a rectangle of sides $c+d$ and $b+a$, with the final result that the area for $\vec{x}=(1,1)$ is $ad-bc$. This formula is important!\\
If we take any random area (think a blob around the origin), we can think of multiplication by $A$ as a transformation of the grid of the cartesian plane itself. Each and every separate point in the blob is transformed just like $(x,y)$ was above. It follows that multiplication by a $2\times2$ matrix $A$ can be generalized as this transformation of the grid. Specifically, if we consider $A\vec{x}=\vec{y}$, then the ratio of the area of our original $\vec{x}$ to the area of our final $\vec{y}$ will be:
\[\frac{\mathrm{area}(\vec{y})}{\mathrm{area}(\vec{x})} = \frac{ad-bc}{1} = ad-bc\]\\
\textbf{Thinking About Signed Area:}\\
Consider if we switch the columns of $A$, that is $A_M=\begin{pmatrix}b&a\\d&c\end{pmatrix}$. Given our $(x,y)$ to multiply, we're going to get a new but very similar parallelogram quadrilateral, with $(0,0)$, $(b,d)x$, $(a,c)y$, and finally $(b,d)x+(a,c)y$. Furthermore, if we calculate our \textbf{determinant}, we get: 
	\[\mathrm{det}(A_M) = bd-ac = -\mathrm{det}(A_M)\]
This means while we get almost the same shape with $A$ and $A_M$, the area of one is going to be negative and the other positive (or both zero)! This makes sense because we are talking about $signed$ area, not unsigned area. The determinant isn't just giving us the area of the parallelogram, but the orientation of this area. For instance, note that $A(x,y)=A_M(y,x)$, and not only are both shapes the exact same -- they have the same determinant!\\ \\ \\
\textbf{Properties of Determinants:}\\
Let's consider the matrix $A=\begin{pmatrix}1&2\\4&8\end{pmatrix}$. It follows that $\mathrm{det}(A)=(1)(8)-(4)(2)=0$. This is important! We can conjecture that for any square matrix $A$, if the columns are linearly dependent, then  $\mathrm{det}(A)=0$.\\
Note these interesting determinant calculations:
\begin{itemize}
	\item $\begin{pmatrix}qa&qc\\b&d\end{pmatrix} = qad-qbc = q(ad-bc) = (q)\mathrm{det}\begin{pmatrix}a&c\\b&d\end{pmatrix}$
	\item $\begin{pmatrix}a_1+a_2&c_1+c_2\\b&d\end{pmatrix} = (a_1+a_2)d-(c_1+c_2)b = \mathrm{det}\begin{pmatrix}a_1&c_1\\b&d\end{pmatrix}+\mathrm{det}\begin{pmatrix}a_2&c_2\\b&d\end{pmatrix}$
\end{itemize}
These rules more generally apply to any of the rows of $A$. Furhtermore, note that we can extend a determinant to any $n \times n$ matrix of any size (even if it quickly gets to be a headache!). See below:
\[\text{If } A=\begin{pmatrix}a_{11}&a_{12}&a_{13}\\a_{21}&a_{22}&a_{23}\\a_{31}&a_{32}&a_{33}\end{pmatrix}\text{, then } \mathrm{det}(A)=a_{11}(a_{22}a_{33}-a_{23}a_{32})-a_{12}(a_{21}a_{33}-a_{23}a_{31})+a_{13}(a_{21}a_{32}-a_{22}a_{31})\]
As a final note, see that for $n\in\mathbb{Z}$, $n\geq2$ the formula for the determinant of an $n\times n$ matrix can be represented recursively as determinants of $(n-1)\times (n-1)$ matrixes. For example, given our formula for $3 \times 3$ matrix $A$, we can show:
\[\mathrm{det}(A)=(a_{11})\mathrm{det}\begin{pmatrix}a_{22}&a_{23}\\a_{32}&a_{33}\end{pmatrix}
-(a_{12})\mathrm{det}\begin{pmatrix}a_{21}&a_{23}\\a_{31}&a_{33}\end{pmatrix}
+(a_{13})\mathrm{det}\begin{pmatrix}a_{21}&a_{22}\\a_{31}&a_{32}\end{pmatrix}\]
In the future, we will explore more about the determinant and its implications.







\end{document}