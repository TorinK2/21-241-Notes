\documentclass[12pt]{amsart}

\addtolength{\hoffset}{-2.25cm}
\addtolength{\textwidth}{4.5cm}
\addtolength{\voffset}{-2.5cm}
\addtolength{\textheight}{5cm}
\setlength{\parskip}{0pt}
\setlength{\parindent}{15pt}

\usepackage{amsthm}
\usepackage{amsmath}
\usepackage{amssymb}
\usepackage{xfrac}
\usepackage[colorlinks = true, linkcolor = black, citecolor = black, final]{hyperref}

\usepackage{graphicx}
\usepackage{multicol}
\usepackage{ marvosym }
\usepackage{wasysym}
\newcommand{\ds}{\displaystyle}


\pagestyle{myheadings}

\setlength{\parindent}{0in}

\pagestyle{empty}

\begin{document}

\thispagestyle{empty}

{\scshape 21-241} \hfill {\scshape \Large Notes} \hfill {\scshape Fall 2020}
\medskip
\hrule
\bigskip

\section*{Change of Basis \& Change of Coordinates (11/30/2020)}
\textbf{Review:}\\
Recall the definitions of similarity and diagonalizability. Matrices $A$ and $B$ are similar if $A = XBX^{-1}$ for some matrix $X$. A matrix is diagonalizable if it is similar to a diagonal matrix, that is $A=X\Lambda X^{-1}$. From our diagonalization, we were able to think about a matrix $A$ as a \textit{change of basis}. The matrices $X$ and $X^{-1}$ swap between standard basis and eigen-basis, and the diagonal matrix multiplies each of the standard basis vectors by a contstant.\\
Also, note that by spectral theorem, for diagonal matrix $S$ we have $S = Q\Lambda Q^{-1}$, where $Q^T=Q^{-1}$. Using this fact, we devised another change of basis system -- Singular Value Decomposition. Given $A = U\Sigma V^T$, we have $U$ transitioning from standard basis to columns of $U$, $\Sigma$ scaling this basis, etc.\\ \\
\textbf{Change of Basis:}\\
In general, let $\{\vec{x}_1, \dots \vec{x}_n\}$ and $\{\vec{y}_1, \dots \vec{y}_n\}$ be bases for $\mathbb{R}^{n}$. How do we transform from $\{\vec{x}_1, \dots \vec{x}_n\}$ to $\{\vec{y}_1, \dots \vec{y}_n\}$? Well, we can use some function composition to get this to work:
\begin{itemize}
	\item We can use $X^{-1}$ to transfer $\vec{x}_i$ to $\vec{e}_i$.
	\item We can use $Y$ to transfer $\vec{e}_i$ to $\vec{y}_i$.
\end{itemize}
So, using the matrix $YX^{-1}$, we can multiply by a vector and change the basis! To go in the other direction, we only need to use $XY^{-1}$. Note that we can transfer to and from bases as so:
\[XY^{-1}YX^{-1}\vec{x} = XX^{-1}\vec{x} = \vec{x}\]
This somewhat justifies what we are doing as sound.\\ \\
\textbf{Defining Coordinates:}\\ \\
Before we address change of coordinates, we need to figure out what exactly coordinates \textit{are}. Let's start with an example coordinate, $(x, y, z)$. This is equivalent to $x\vec{e}_1+y\vec{e}_2+z\vec{e}_3$. So, our coordinates are generally defined with respect to the standard basis! However, we can now take coordinates with respect to \textit{any} basis. Given our basis $\{\vec{x}_1, \vec{x}_2, \vec{x}_3\}$ making up the columns of matrix $X$, we see that given coordinate $(a, b, c)$, and $X^{-1}(a,b,c) = (a', b', c')$, we find:
\[(a,b,c) = a\vec{e}_1 + b\vec{e}_2 + v\vec{e}_3 = a'\vec{x}_1 + b'\vec{x}_2 + c'\vec{x}_3\]
This means  the vector with coordinates $(a', b', c')$ with respect to the $x$-basis is equivalent to the vector $(a,b,c)$ with respect to the standard basis. This is because of the fact that we can use $X^{-1}$ to transfer $\vec{x}_i$ to $\vec{e}_i$. Thus, we call $X^{-1}$ our ``change of coordinates" matrix in this system.\\
Similarly to change of basis, we can use function decomposition to determine change of coordinates. The vector $\vec{v}$ with coordinates $(a_1, \dots, a_n)$ with respect to the standard basis is equivalent to the vector with the coordinates $X^{-1}\vec{v}$ with respect to the $x$-basis. The vector $\vec{v}$ with coordinates $(a_1, \dots, a_n)$ with respect to the $y$-basis is the same as the vector with coordinates $Y\vec{v}$ with respect to the standard basis. So, to transfer from the $y$-basis to the $x$-basis, we simply ``go through" the standard basis by multiplying via ``change of coordinates" matrix $X^{-1}Y$.\\ \\
\textbf{Cameras and Perspectives:}\\
We can take this cool change of coordinates process and use it to think about looking at an image from a variety of different perspectives. This is what we call \textit{perspective rendering}. The problem in mind is specifically this:
\begin{itemize}
	\item We have a thing that we want to render, with coordinate on the tree $\vec{p}$.
	\item We have a \underline{camera} at a specific point in space, at point $\vec{x}$.
	\item We have a \underline{sensor array} in front of this point, which we call the ``image plane."
\end{itemize}
So, light comes from $\vec{p}$, through the image plane, to get to the pont $\vec{x}$. The image plane is really an array of pixels. At this point, we will redefine our conventions. Thinking about this image plane, the origin is the top left corner, while $x$ and $y$ go to the right and down respectively. Now, we have point $p\in\mathbb{R}^3$ on the shape, what are the coordinates of its corresponding pixel $q$ in the image plane?\\
Well, first we define a basis for the image plane. Let $\vec{a}_1$ go horizontal along the image plane, and $\vec{a}_2$ go vertical along the image plane. These will each be exactly one pixel long, and will be the first two vectors in our basis. Now, $\vec{a}_3$ will be the vector from the camera to the top left corner of the image plane.\\
Using this new basis and our change of coordinate system, we will be able to actually implement perspective rendering next lecture!

\end{document}